\section{Určování polohy hrany jako průchod druhé derivace obrazové funkce}
Jakou výhodu přináší určování polohy hrany jako průchodu druhé derivace obrazové funkce nulovou hladinou? 
Napište, v jakých hranových detektorech se této výhody využívá a jak.

Určování polohy hrany pomocí průchodu druhé derivace nulou umožňuje její velmi přesnou lokalizaci na šířku jediného 
pixelu. Tato metoda je robustnější vůči šumu a poskytuje uzavřené hrany, což je výhodné pro další zpracování obrazu.

Této výhody využívá především Marr-Hildrethův detektor hran. Ten nejprve aplikuje na obraz filtr známý jako Laplacián 
Gaussovy funkce (LoG), čímž dojde k vyhlazení a zároveň výpočtu druhé derivace. V takto zpracovaném obraze jsou pak 
hrany detekovány právě v místech, kde funkční hodnoty procházejí nulou.