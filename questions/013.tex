\section{Geometrická optika}
Fungování objektivu fotoaparátu se obvykle na praktické úrovni vysvětluje teorií geometrické optiky. Za jakých 
předpokladů se může být zjednodušený model geometrické optiky použít? Podotýkám, že nejbližší další model v řadě 
složitějších fyzikálních modelů je model vlnové optiky.

Zjednodušený model geometrické optiky lze použít za předpokladu, že vlnová délka světla je zanedbatelně malá ve srovnání 
s rozměry optických prvků, se kterými interaguje.

Tato klíčová podmínka ($\lambda \ll d$) umožňuje ignorovat vlnové jevy, jako je ohyb (difrakce) a interference.