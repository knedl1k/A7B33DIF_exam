\section{Licenční smlouva, výhradní license, třetí osoba, odměna, omezení licence, odstoupení od smlouvy, školní dílo}
Licenční smlouva §~46, Ne/Výhradní licence §~47, Třetí osoba §~48, Odměna §~49, Omezení licence §~50, 
Odstoupení od smlouvy a zánik licence, Školní dílo §~60.

Pokud autor poskytuje někomu jinému (nabyvateli) právo dílo užít, pak tak činí pomocí licence. Je zde smluvním stranám 
dávána značná volnost a vyjma samotné písemné licence není vyžadována smlouva písemná.

Zjednodušeně lze říci, že nevýhradní licence umožňuje dále vykonávat veškerá práv spojených s dílem, taková licence 
nemusí mít písemnou formu. Naopak písemná forma je vyžadovaná u licence výhradní, ta autorovi neumožňuje dále vykonávat 
práva, k nimž udělil výhradní licenci.

Není-li dále stanoveno jinak, musí být ve smlouvě dohodnuta výše odměny nebo v ní musí být alespoň stanoven způsob 
jejího určení.

Licence může být omezena na jednotlivé způsoby užití díla; způsoby užití díla mohou být omezeny rozsahem, zejména co do 
množství, místa nebo času.

Nevyužívá-li nabyvatel výhradní licenci vůbec nebo využívá-li ji nedostatečně a jsou-li tím značně nepříznivě
dotčeny oprávněné zájmy autora, může autor od smlouvy odstoupit. Autor může písemně odstoupit od smlouvy,
jestliže jeho dosud nezveřejněné dílo již neodpovídá jeho přesvědčení a zveřejněním díla by byly značně
nepříznivě dotčeny jeho oprávněné osobní zájmy.

Škola nebo školské či vzdělávací zařízení jsou oprávněny požadovat, aby jim autor školního díla z výdělku jím dosaženého 
v souvislosti s užitím díla či poskytnutím licence podle odstavce 2 přiměřeně přispěl na úhradu nákladů, které na 
vytvoření díla vynaložily, a to podle okolností až do jejich skutečné výše. Není-li sjednáno jinak, může autor školního 
díla své dílo užít či poskytnout jinému licenci, není-li to v rozporu s oprávněnými zájmy školy nebo školského či 
vzdělávacího zařízení.