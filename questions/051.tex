\section{Filtrace náhodného aditivního šumu v obraze}
Uvažujte filtraci náhodného aditivního šumu v obraze. Odhad správné hodnoty se může počítat jako aritmetický průměr $n$ 
zašuměných hodnot. Kolikrát se po filtraci zmenší hodnota šumu vyjádřená směrodatnou odchylkou $\sigma$? Vysvětlete, 
jaký je statistický princip poklesu šumu (nápověda: centrální limitní věta).

Při zprůměrování $n$ nezávislých zašumělých hodnot se směrodatná odchyulka šumu $\sigma$ zmenší $\sqrt{n}$-krát.

Statistický princip tohoto jevu vychází z pravidel pro práci s náhodnými veličinami a je úzce spjat s \bb{Centrální 
limitní větou}. Když sčítáme nezávislé náhodné veličiny (jednotlivé hodnoty šumu), jejich \bb{rozptyly} (variance) se 
sčítají, nikoli směrodatné odchylky. Rozptyl průměru z $n$ měření je tedy $n$-krát menší než rozptyl jednoho měření. 
Vzhledem k tomu, že směrodatná odchylka je odmocnina z rozptylu, její pokles je $\sqrt{n}$. Průměrováním se tak náhodné 
kladné a záporné odchylky šumu efektivně \enquote{vyruší} a odhadovaná hodnota se stále více blíží skutečné, nezašuměné 
hodnotě signálu.