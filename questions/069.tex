\section{Kompresní metody specializované na obrazy}
Metody komprese se používají i pro jednorozměrné signály. I obraz je možné reprezentovat jako jednorozměrný 
signál, což například uděláme, když obraz zazipujeme (použije se algoritmus LZW pracující se slovníkem). U kompresních 
metod specializovaných na obrazy můžeme dosáhnout vyšší komprese. Proč? Pro vysvětlení použijte pojem redundance dat. 
(Odpověď dává také odpověď na přirozenou otázku: Čím se liší komprese obrázků od komprese signálů?).

Kompresní metody specializované na obrazy dokáží využít prostorovou redundanci dat, což obecné metody neumí. To znamená,
že faktu, že sousední pixely mají velmi často podobnou barvu a jas.