\section{Skladebný princip rámu obrazu}
Vysvětlete skladebný princip rámu obrazu (funkci rámu obrazu).

Princip rámu obrazu, často nazývaný \enquote{rám v rámu}, je kompoziční technika, při které se využívají 
prvky přímo ve fotografované scéně k vytvoření vnitřního, přirozeného orámování hlavního motivu. Nejedná se tedy o 
fyzický rám pověšeného obrazu, ale o kreativní využití popředí nebo okolí scény k posílení celkového vizuálního sdělení.