\section{Příklad - vztah pro vyhlazování histogramu}
Zapište vztah pro vyhlazování histogramu $h_i$, $i=0,\dots,255$ pomocí klouzavého průměru pro okno o šířce $2K+1$ s 
reprezentativní hodnotou okna uprostřed.
\begin{align}
    h^\prime (i) = \frac{1}{2K+1}\sum_{j=i-K}^{i+K}h(j),
\end{align}
kde $h^\prime(i)$ je nová, vyhlazená hodnota histogramu v pozici $i$; $h(j)$ jsou původní hodnoty histogramu;
$2K+1$ je celková šířka symetrického vyhlazovacího okna (např. pro $K=2$ je šířka okna 5).

Operace tedy nahradí každou hodnotu průměrem sebe sama a svých K levých a K pravých sousedů, čímž se potlačí šum a 
vyhladí lokální extrémy.