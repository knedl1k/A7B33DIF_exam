\section{Úloha segmentace obrazu}
Formulujte úlohu segmentace obrazu (vstup, výstup, co má udělat?). Napište, na co se segmentace používá při 
zpracování digitálních fotografií.

Segmentace obrazu je proces, který rozděluje digitální obraz na několik smysluplných, nepřekrývajících se oblastí neboli 
segmentů. Cílem je zjednodušit obraz a seskupit pixely, které k sobě patří, aby bylo možné dále analyzovat jen konkrétní 
objekty.
\begin{itemize}
    \item \bb{Vstup}: digitální obraz
    \item \bb{Cíl}: Najít a ohraničit souvislé oblasti v obraze. Pixely uvnitř jednoho segmentu by si měly být co 
    nejvíce podobné (např. barvou, texturou), zatímco sousední segmenty by se měly co nejvíce lišit.
    \item \bb{Výstup}: Maska nebo mapa segmentů. Je to obraz stejné velikosti jako vstup, kde hodnota každého pixelu 
    neudává barvu, ale identifikační číslo segmentu, do kterého patří.
\end{itemize}
Využívá se třeba na výběr objektů pomocí \enquote{kouzelné hůlky}, rozmazání pozadí...