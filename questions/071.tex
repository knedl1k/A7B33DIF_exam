\section{Princip ztrátové komprese obrázku pomocí lineárních integrálních transformací}
Vysvětlete princip ztrátové komprese obrázků pomocí lineárních integrálních transformací. Vyjmenujte dvě takové 
metody a naznačte jejich princip. Proč se pro obrazy používají jiné metody komprese než pro posloupnosti?

\begin{enumerate}[a)]
    \item Diskrétní kosinová transformace - Obraz se nejprve rozdělí na malé bloky (typicky $8 \times 8$ pixelů) a na 
    každý z nich se samostatně aplikuje DCT.
    \item Vlnková transformace - Na rozdíl od DCT, která pracuje s pevnými bloky, vlnková transformace analyzuje celý 
    obraz najednou v různých úrovních rozlišení.
\end{enumerate}