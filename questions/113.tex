\section{Skladebný princip rytmu a princip těžiště}
Vysvětlete skladebný princip rytmu a princip těžiště.

Princip rytmu v kompozici je založen na pravidelném opakování nebo střídání podobných prvků. Podobně jako v hudbě, kde 
rytmus tvoří opakující se údery, ve vizuálním umění (fotografie, malba, design) vytváří rytmus opakování linií, tvarů, 
barev nebo tónů.

Princip těžiště říká, že každá správná kompozice by měla mít jeden hlavní bod zájmu (dominantní prvek), který jako první 
přitáhne pozornost diváka. Toto "těžiště" funguje jako vizuální kotva, kolem které je postaven zbytek obrazu. Není to 
nutně geometrický střed snímku - naopak, často bývá umístěno mimo něj (např. podle pravidla třetin).