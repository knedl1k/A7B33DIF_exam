\section{Definice přímé a inverzní jednorozměrné Fourierovy transformace}
Napište definiční vztah pro přímou a inverzní jednorozměrnou Fourierovu transformaci. Vyjádřete neformálně 
princip a význam Fourierovy transformace.
\begin{align}
    F(\xi) &= \int_{-\infty}^{\infty}f(t)e^{-2 \pi \iu \xi t} \dif t \, (\text{Přímá FT})\\
    f(t) &= \int_{-\infty}^{\infty}F(\xi)e^{2 \pi \iu \xi t} \dif \xi \, (\text{Inverzní FT})
\end{align}
Fourierova transformace funguje na principu rozkladu složitého signálu ($f(t)$) na součet jednoduchých sinusových vln o 
různých frekvencích ($\xi$). Lze si to představit jako hudební sluch, který v komplexním akordu rozpozná jednotlivé 
tóny.

Význam: Výsledkem transformace je frekvenční spektrum ($F(\xi)$), které nám ukáže, jaké frekvence jsou v původním 
signálu obsaženy a s jakou intenzitou.