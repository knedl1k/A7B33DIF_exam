\section{Barevný prostor}
Co je barevný prostor? Jak je definován? Uvažujte pro jednoduchost barevný prostor barevných senzorů v lidském 
oku. Nezapomeňte uvést souvislost s barevným metamerismem.

V podstatě jde o souřadnicový systém, kde má každá barva své přesné místo. Pro lidské oko je tento fundamentální barevný 
prostor definován odezvou jeho tří typů barevných senzorů - čípků (S, M, L). Každá barva, kterou jsme schopni vnímat, 
tak odpovídá jedinečné trojici hodnot (S, M, L) podle toho, jak silně je daným světlem každý typ čípku stimulován.

Barevný metamerismus je klíčový jev, kdy dva světelné zdroje s rozdílným fyzikálním spektrem vnímáme jako naprosto 
identickou barvu. K tomu dochází právě proto, že tyto dva fyzikálně odlišné signály dokáží v našich čípcích vyvolat 
úplně stejnou trojici odezev (S, M, L).