\section{Spojitá obrazová funkce}
Vysvětlete pojem spojitá obrazová funkce $f (x, y)$ nebo $f (x, y, t)$. Vysvětlete, co jsou parametry $x, y, t$. Uveďte 
několik příkladů reálných obrazových funkcí sejmutých s pomocí různých fyzikálních principů. Hodnota funkce $f$ tedy 
bude odpovídat různým fyzikálním veličinám.

Spojitá obrazová funkce, popisovaná jako $f(x,y)$ pro statické obrazy nebo $f(x,y,t)$ pro dynamické scény, je 
matematický model reálného světa. Parametry $x$ a $y$ představují spojité prostorové souřadnice v rovině obrazu, zatímco 
$t$ reprezentuje čas. Hodnota samotné funkce $f$ pak odpovídá intenzitě určité fyzikální veličiny, která je v daném bodě 
měřena. Například u klasické fotografie je to \textbf{jas}, u termovize \textbf{teplota}, u rentgenového snímku 
\textbf{míra pohlcení záření} tkání, nebo u dat ze sonaru \textbf{vzdálenost} od senzoru.