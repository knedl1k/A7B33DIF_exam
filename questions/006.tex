\section{Co je to kvantování obrazu? Jak a v jakém zařízení se kvantování realizuje? Kolik kvantizačních úrovní zhruba 
rozliší u monochromatického obrazu člověk? Co je v obraze patrné, když je kvantizačních úrovní méně, než by mělo být?}.
\begin{itemize}
  \item{Co je to kvantování obrazu? Jak a v jakém zařízení se kvantování realizuje?\\}
    Jde o diskretizaci barev, digitální kamara, např., diskretizuje spojité spektrum barev, které vnímáme, aby je mohla uložit do bitů v paměti. Dělá to tím způsobem že barvě ze spektra přiřadí nejbližší barvu z palety.
  \item{Kolik kvantizačních úrovní zhruba rozliší u monochromatického obrazu člověk?\\}
    Člověk rozliší okolo 16 úrovní šedi.
  \item{Co je v obraze patrné, když je kvantizačních úrovní méně, než by mělo být?\\}
    Když je úrovní málo, tak je výsledný obraz plný jednobarevných ploch, vypadá podobně jako vrstevnice na mapě.
\end{itemize}
