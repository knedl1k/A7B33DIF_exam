\section{Symetrie, asymetrie}
Vysvětlete následující kompoziční schemata: symetrie, asymetrie.

Symetrie je skladebný princip, který znamená pravidelné rozmístění prvků kolem středu nebo kolem některé osy.

Asymetrie je opak symetrie a znamená rozložení kontrastních kompozičních prvků tak, aby jejich celková skladba působila 
vyváženě.