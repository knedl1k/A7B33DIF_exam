\section{Digitalizace dvojrozměrného obrazu}
Uvažujte digitalizaci dvojrozměrného obrazu. Zde se stejně jako při digitalizaci jednorozměrného signálu 
stanovuje vzdálenost ekvidistantních vzorků podle Shannonovy věty o vzorkování. Pro dvojrozměrné obrazy je potřebné 
navíc ke stanovení vzdálenosti mezi vzorky (což se řeší podobně jako u jednorozměrného signálu) vyřešit další 
záležitost. Jakou? Jak se záležitost typicky řeší a jaké výhody či nevýhody tato řešení mají? Poznamenávám, že se neptám 
na kvantování.

Při digitalizaci dvojrozměrného obrazu je kromě vzdálenosti vzorků nutné vyřešit jejich geometrické uspořádání, tedy 
zvolit typ vzorkovací mřížky.

Nejčastěji se používá čtvercová mřížka, a to především pro její snadnou technickou realizaci a přímou kompatibilitu s 
hardwarem (senzory, displeje) i algoritmy (např. Fourierova transformace). Její klíčovou nevýhodou je však 
nejednoznačnost při měření vzdáleností a sousednosti, protože diagonální soused pixelu je dál než jeho přímý soused 
(tzv. problém metriky).

Teoreticky výhodnější hexagonální mřížka tento problém řeší, jelikož v ní má každý pixel všechny sousedy ve stejné 
vzdálenosti. V praxi se ovšem téměř nevyužívá kvůli nedostatku hardwarové podpory a větší složitosti při implementaci 
standardních operací.