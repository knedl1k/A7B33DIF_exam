\section{Hloubka zaostření}
Vysvětlete pojem hloubka zaostření u optického objektivu. Jaký (obvykle ovladatelný) parametr objektivu umožňuje měnit 
hloubku zaostření?

Hloubka ostrosti (ano, takto se to správně jmenuje) je rozsah vzdáleností ve snímané scéně, který se na výsledné 
fotografii jeví jako přijatelně ostrý. Nejde tedy o jedinou rovinu, ale o celou \enquote{zónu ostrosti} před i za 
zaostřeným objektem. Hlavním ovladatelným parametrem objektivu pro změnu hloubky ostrosti je \textbf{clona} (aperture), 
jejíž velikost se udává \textbf{clonovým číslem} (f-number).
\begin{itemize}
    \item Nízké clonové číslo (např. f/1.8, \enquote{otevřená clona}) znamená malou hloubku ostrosti. Ostrý je jen 
    hlavní objekt, zatímco pozadí a popředí jsou rozmazané (využití u portrétů).
    \item Vysoké clonové číslo (např. f/16, \enquote{zavřená clona}) znamená velkou hloubku ostrosti. Ostrá je velká 
    část scény od popředí až po horizont (využití v krajinářské fotografii).
\end{itemize}