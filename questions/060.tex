\section{Marrův přístup k detekci hran využívá hledání průchodu druhé derivace obrazové funkce nulou. Při výpočtu 
derivace se s výhodou pro potlačení vlivu šumu používá konvoluce (rozmazání) gaussovským filtrem $g$. Druhá derivace 
takové operace nechť je označena $\nabla^{2}d = \nabla^{2}(f*g) = (\nabla^{2}f)*g = f*(\nabla^{2}g)$. Metoda využívá 
vtipný trik (obejde derivaci obrazové funkce $f$). Prosím, abyste ho použili a pokračovali v předchozím odvození. Díky 
jakým vlastnostem použitých operací lze trik použít?}