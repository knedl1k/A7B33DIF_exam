\section{Přístroje a metody měření světla}
Popište přístroje a metody měření světla ve fotografii.

Světlo ve fotografii se měří přístrojem zvaným expozimetr, který je buď integrovaný přímo ve fotoaparátu, nebo se 
používá jako samostatné externí zařízení. Interní expozimetr nejčastěji měří světlo odražené od fotografované scény, 
což je pohodlná, ale méně přesná metoda. Externí ruční expozimetry naproti tomu umožňují měřit světlo dopadající na 
scénu, což zaručuje přesnější výsledky nezávislé na barvě objektu. Cílem obou metod je stanovit správné hodnoty clony, 
expozičního času a ISO pro dosažení technicky správně exponovaného snímku.