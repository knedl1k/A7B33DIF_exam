\section{Egyptský trojúhelník, zlatý řez}
Vysvětlete pojmy: egyptský trojúhelník, zlatý řez.

Zlatý řez je poměr, který se v umění a fotografii pokládá za ideální proporci mezi různými délkami. Zlatý řez vznikne 
rozdělením úsečky na dvě části tak, že poměr větší části k menší je stejný jako poměr celé úsečky k větší části.

Při jednoduché konstrukci tvaru pyramidy, pokud nemáme k dispozici přesné rozměry v poměru k originálním stavbám, lze 
využít tzv. Egyptský (pravoúhlý) trojúhelník podle zásady zlatého řezu - poměr stran tohoto pravoúhlého trojúhelníku je 
3:4:5.
