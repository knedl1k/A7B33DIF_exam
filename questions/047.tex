\section{Použití ekvalizace histogramu na šedotónový obrázek}
Uvažujte šedotónový obrázek. Ekvalizace histogramu se využívá pro zvýšení kontrastu lepším využitím jasové 
stupnice. Zvyšuje ekvalizace histogramu množství informace v obrazu, pokud bychom množství informace měřili Shannonovou 
entropií? Vysvětlete a uveďte příklady.

Ano, prakticky ve všech případech ekvalizace histogramu zvyšuje množství informace v obraze, pokud ji měříme Shannonovou 
entropií. Důvodem je, že Shannonova entropie je matematicky maximální pro dokonale plochý (uniformní) histogram, kde 
mají všechny úrovně jasu stejnou pravděpodobnost výskytu. Cílem ekvalizace histogramu je právě transformovat původní, 
nerovnoměrný histogram tak, aby se tomuto ideálnímu plochému rozložení co nejvíce přiblížil.

Příklad: Vezměme si podexponovaný, nízkokontrastní snímek. Jeho histogram bude úzký a špičatý, soustředěný v tmavých 
tónech, a bude mít nízkou entropii. Po ekvalizaci se tyto hodnoty \enquote{roztáhnou} přes celý dynamický rozsah, 
histogram se zploští a jeho entropie se tím zákonitě zvýší.