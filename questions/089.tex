\section{Právo autorské: dílo, autor, spoluautoři, vznik práva}
Předmět práva autorského: Dílo §~2, Autor §~5, Spoluautoři §~8. Vznik práva autorského §~9.

Dílo, které je předmětem ochrany autorského práva je §~2 definováno jako jakkoli vnímatelné dílo, které je 
výsledkem jedinečné tvůrčí činnosti fyzické osoby.

Autorem je fyzická osoba, která dílo vytvořila, nebo v případě souborných děl (resp. databází) uspořádala či vybrala, 
což je ale nutno činit \enquote{tvůrčím způsobem}, protože jinak by nešlo o autora, ale plagiátora.

Ze zákona vyplývá, že spoluautorství je možné pouze ve stejném oboru lidské činnosti (spoluautory tedy nejsou např. 
fotograf a spisovatel, protože jde o dvě různá autorská díla, která mohou být spojena v majetkových, ale nikoli 
osobnostních právech).

Právo vzniká při tvorbě díla, při stisknutí spouště. Kontinentální a v poslední době i anglo-americké právo přiznává 
vznik tohoto práva automaticky se vznikem díla, tzn. autor nemusí nic dělat a právo samovolně vzniká při vzniku díla.