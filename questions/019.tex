\section{Proč vidíme některé objekty barevně}
Proč vidíme některé objekty barevně? Uvažujte např. jednu čerstvě ustřiženou červenou růži. Vysvětlete, proč 
vidíme stonek zeleně a květ červeně.

Barevné vidění objektů je výsledkem interakce světla s jejich povrchem. Běžné bílé světlo (např. sluneční) obsahuje celé 
spektrum barev. Povrch objektu následně některé vlnové délky tohoto světla pohltí a jiné odrazí. Naše oko zachytí tyto 
odražené vlnové délky a mozek je interpretuje jako příslušné barvy. Objekt tedy nemá barvu sám o sobě, ale pouze odráží 
určitou část světla, které na něj dopadá.