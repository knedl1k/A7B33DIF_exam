\section{Dvojrozměrná Fourierova transformace}
Vysvětlete, co je dvojrozměrná Fourierova transformace, její rozdíl od jednorozměrné (můžete definičním vzorcem 
nebo neformálně) a jak se používá ve zpracování obrazu.

Dvojrozměrná Fourierova transformace (2D FT) je rozšíření 1D FT pro dvourozměrná data, typicky obrazy. Místo rozkladu 
signálu na 1D sinusovky rozkládá celý obraz na součet 2D sinusových mřížek (gratings), které mají různou frekvenci 
(hustotu), amplitudu (kontrast) a orientaci. 

Výsledné 2D frekvenční spektrum je v podstatě \enquote{mapa} frekvencí v obraze. Střed spektra odpovídá nízkým 
frekvencím (pomalé změny barev, celkové tvary), zatímco body dále od středu reprezentují vysoké frekvence (ostré hrany, 
detaily, šum).

Hlavní využití 2D FT je pro filtrování obrazu ve frekvenční doméně. Je to často mnohem efektivnější a elegantnější 
metoda než prostorové konvoluce.