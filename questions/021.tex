\newpage
\section{Senzory pro barevné vidění v lidském oku}
Jaké senzory jsou v lidském oku pro barevné vidění? Nakreslete zhruba citlivost jednotlivých senzorů grafem, 
kde na vodorovné ose bude vlnová délka kvantifikovaná v nanometrech [nm] a na svislé ose relativní citlivost v rozshu od 
0 do 1.

\begin{figure}[ht]
    \centering
    \begin{tikzpicture}[x=0.03cm, y=3cm] % Měřítko pro osy

        % --- axies ---
        \draw[-Latex, thick] (370,0) -- (700,0) node[midway, below=1.5em] {Vlnová délka (nm)};
        \draw[-Latex, thick] (380,0) -- (380,1.2) node[anchor=south, midway, sloped, above=2em] {Normalizovaná citlivost};

        % --- Markings on axies ---
        \foreach \x in {400, 500, 600}
            \draw (\x, 3pt) -- (\x, -3pt) node[below] {\x};
        \foreach \y in {0.5, 1}
            \draw (377, \y) -- (383, \y) node[left=4pt] {\y};
        \node[left=4pt] at (390,-0.1) {0};

        % --- S-chip (blue) ---
        \draw[blue, very thick, rounded corners=3pt] 
            plot coordinates {(380,0.05) (400,0.35) (420,0.98) (440,0.90) (460,0.50) (500,0.10) (520,0.02)};
        \node[above, blue, font=\small] at (420,0.98) {420};

        % --- Rods (grey) ---
        \draw[black!60, very thick, rounded corners=3pt] 
            plot coordinates {(410,0.1) (450,0.70) (475,0.95) (498,1.0) (520,0.75) (550,0.30) (580,0.05)};
        \node[above, black!75, font=\small] at (498,1.0) {498};

        % --- M-chip (green) ---
        \draw[green!50!black, very thick, rounded corners=3pt] 
            plot coordinates {(450,0.05) (480,0.40) (510,0.85) (534,1.0) (560,0.85) (580,0.50) (620,0.08)};
        \node[above, green!50!black, font=\small] at (534,1.0) {534};

        % --- L-chip (red) ---
        \draw[red, very thick, rounded corners=3pt] 
            plot coordinates {(470,0.05) (500,0.25) (530,0.75) (564,1.0) (600,0.60) (630,0.20) (650,0.02)};
        \node[above, red, font=\small] at (564,1.0) {564};

    \end{tikzpicture}
\end{figure}
Senzory pro barevné vidění v lidském oku jsou fotoreceptorické buňky zvané \textbf{čípky}, které se nacházejí v sítnici. 
Existují tři typy čípků (na grafu modrá, zelená, červená), z nichž každý je citlivý na jinou část světelného spektra, 
což nám umožňuje vnímat kompletní škálu barev.

Šedá křivka navíc zobrazuje citlivost \textbf{tyčinek}, druhého typu senzorů, které jsou mnohem citlivější na světlo a 
umožňují nám nebarevné vidění v šeru a tmě.