\section{Shannonova věta o vzorkování}
Formulujte Shannonovu (též Nyquistovu, Kotelnikovu) větu o vzorkování pro jednodušší případ jednorozměrného 
signálu. Vysvětlete (stačí neformálně, obrázek pomůže), jak se věta o vzorkování dokazuje (nápověda: frekvenční 
spektra).

Shannonova věta o vzorkování říká, že spojitý signál, který neobsahuje žádné frekvence vyšší než $f_{max}$, lze plně 
zrekonstruovat z jeho diskrétních vzorků, pokud byla vzorkovací frekvence $f_s$ alespoň dvakrát větší než tato maximální 
frekvence.

Matematicky to vyjadřuje podmínka: 
\begin{align}
    f_s > 2 f_{max}
\end{align}
Vzorkování signálu v čase způsobí, že se jeho původní frekvenční spektrum začne v doméně frekvencí 
\bb{periodicky opakovat} s rozestupem rovným vzorkovací frekvenci $f_s$. 

Pokud je podmínka splněna, tyto repliky spektra se nepřekrývají. Původní signál pak lze dokonale obnovit jednoduchým 
odfiltrováním základního spektra pomocí ideální dolní propusti.

Pokud podmínka splněna není, repliky se překryjí, dojde k nevratnému zkreslení spektra (jev zvaný aliasing) a signál 
již zrekonstruovat nelze.