\section{Barevný rozsah zařízení}
Co znamená barevný rozsah určitého snímacího nebo zobrazovacího zařízení? Jak barevný rozsah souvisí s barevným 
trojúhelníkem? Srovnejte kvalitativně barevný rozsah kvalitního barevného filmu a rozsah levné barevné počítačové 
tiskárny.

Barevný rozsah je kompletní sada všech barev, kterou je dané zařízení schopno zaznamenat (film) nebo 
reprodukovat (tiskárna). Tento rozsah se vizualizuje jako barevný trojúhelník v diagramu všech lidským okem viditelných 
barev; čím větší je plocha tohoto trojúhelníku, tím více barev zařízení umí. Při srovnání má kvalitní barevný film 
mnohem větší barevný rozsah než levná barevná tiskárna. Film dokáže díky bohatým chemickým barvivům zachytit velmi syté 
a živé barvy, které se blíží realitě. Levná tiskárna je naopak omezena vlastnostmi a čistotou CMYK inkoustů, které 
nedokáží na papíře vytvořit tak syté odstíny, a její gamut je proto výrazně menší.