\section{Vnímání a užití světla}
Vysvětlete následující pojmy: Vnímání a užití světla člověkem. Význam absorpce a odrazu světla předměty. 
Funkce stínu. Technická a výtvarná funkce světla.

Vnímání je subjektivním odrazem objektivní reality v našem vědomí prostřednictvím receptorů a umožňuje základní 
orientaci v prostředí. Člověk světlo užívá jak prakticky - abychom viděli, tak výtvarně, abychom s ním něco vytvořili.

Absorpce světla je fyzikální jev, při němž dochází k zeslabování intenzity záření. To znamená, že světelná energie se 
mění na jiné formy energie prostředí. Odraz světla je jev, při kterém se světlo vrací do původního prostředí.

Stín je oblast, kde přímé světlo nezasahuje kvůli překážce mezi zdrojem světla a povrchem. Stín může být použit k 
vytvoření hloubky a kontrastu ve vizuálním umění

Technická funkce světla spočívá v poskytnutí dostatečného osvětlení pro viditelnost a orientaci. Výtvarná funkce světla 
spočívá ve vytváření atmosféry a estetického dojmu pomocí barev, stínů a kontrastů.