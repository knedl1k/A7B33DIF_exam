\section{Lokální a globální zpracování.}
\begin{itemize}
  \item Diskutujte stručně rozdíl mezi lokálním a globálním přístupem v analýze obrazu. Uveďte výhody a nevýhody 
    obojího.
    \begin{itemize}
      \item{globální analýza obrazu:\\}
        Dívám se na obraz jako na celek, vidím vše najednou, a jako celek ho analyzuji.
        Při tomhle pohledu vidím ho jako celek, jako lidi tak analyzujeme co je na obraze zobrazeno, nedíváme se ale na detaily.
      \item{lokální analýza obrazu:\\}
        Obraz analyzu v různých lokálních oknech, tímto způsobem analyzují obraz stroje.
        Při lokální analýze může být dost těžké zjistit, co je na obrazu jako celek vyzobrazeno, zase ale vidíme detaily co by nám jinak unikly.
    \end{itemize}
  \item{Uveďte se stručným komentářem dva příklady lokálních operací:\footnotemark{}\\}
    Bodová retuš, tj. např. mazání pih, lokální zesvětlení/ztmavení, cenzura nějakých elementů (rozostření obličeje, rozpixelování textu).
  \item{Uveďte se stručným komentářem dva příklady globálních operací:\footnotemark[\value{footnote}]{}\\}
    White-balance (curves, úprava barev), otáčení obrazu, ořezávání (změna formátu obrazu).
    \footnotetext{nejsem si úplně jist co se tímto myslí ale rozumím této otázce ve smyslu úpravy obrazů}
\end{itemize}.
