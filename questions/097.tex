\section{Složení/kvalita světla}
Vysvětlete pojem složení/kvalita světla a možnosti jeho regulace ve fotografii.

Různé světelné zdroje mají různou barvu, kterou mozek automaticky koriguje, ale fotoaparát ji zaznamenává přesně.
\begin{itemize}
    \item Teplé světlo (oranžové): Svíčka, žárovka, východ/západ slunce.
    \item Neutrální světlo (bílé): Polední slunce, studiový blesk.
    \item Studené světlo (namodralé): Stín za jasného dne, zatažená obloha.
\end{itemize}

Tvrdost světla - popisuje charakter přechodů mezi světlem a stínem.
\begin{itemize}
    \item Tvrdé světlo: Pochází z malého, přímého zdroje (např. přímé slunce v poledne, holý blesk). Vytváří ostré, 
    jasně ohraničené stíny a vysoký kontrast.
    \item Měkké světlo: Pochází z velkého, rozptýleného (difúzního) zdroje (např. zatažená obloha, světlo z okna na 
    sever, blesk se softboxem). Vytváří jemné, pozvolné stíny.
\end{itemize}

Regulovat lze přímo ve fotoaparátu pomocí clony, expozice, ISO, či vyvážení bílé. Také lze regulovat přímo prostředím
- zatáhnout zaclonu, ztlumit světlo...
