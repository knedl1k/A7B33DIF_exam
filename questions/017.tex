\section{Radiální zkreslení objektivu}
Vysvětlete, co je to radiální zkreslení objektivu. Jak se v sejmutém obraze projevuje a jak se opravuje?

Radiální zkreslení je běžná optická vada objektivu, která způsobuje, že rovné linie ve skutečnosti se na fotografii 
zobrazí jako zakřivené. Tento efekt je nejvýraznější směrem k okrajům snímku, zatímco střed obrazu zůstává téměř 
neovlivněn.

Oprava probíhá digitálně aplikací matematické transformace, která je inverzní k danému zkreslení. Na základě předem 
známých kalibračních koeficientů pro konkrétní objektiv se pixely v obraze přesunou na své správné, nezkreslené pozice, 
čímž se zakřivené linie opět narovnají.