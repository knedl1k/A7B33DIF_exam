\section{Barevná kalibrace počítačového monitoru}
Co je barevná kalibrace počítačového monitoru? Proč a jak se monitory barevně kalibrují?

Barevná kalibrace je proces nastavení monitoru tak, aby zobrazoval barvy co nejpřesněji a v souladu s definovaným 
standardem (např. sRGB).
Cílem je zaručit přesnost a konzistenci barev, aby obrázek vypadal stejně na různých zařízeních a odpovídal tvůrčímu 
záměru. Monitory se kalibrují sondou, která se přiloží přímo na obrazovku. Tato sonda postupně měří sérii zobrazených 
barevných políček a porovnává je s tím, jaké by měly být podle standardu.