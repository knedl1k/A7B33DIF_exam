\section{Princip ostřední obrazu}
Představte si, že máte k dispozici již sejmutý digitální obraz. Vysvětlete princip ostření obrazu (neptám se 
globální úpravu jasové stupnice podle histogramu). Co je cílem ostření? V jakých situacích se ostření používá?

Cílem ostření je zvýšit vnímanou ostrost obrazu zvýrazněním hran a jemných detailů. Princip nejčastěji spočívá v použití 
tzv. neostré masky (unsharp masking), kdy se od originálního obrazu odečte jeho rozmazaná verze, čímž se izolují právě 
hrany. Tyto izolované hrany jsou následně zesíleny a přičteny zpět k původnímu obrazu, což vede ke zvýšení kontrastu v 
jejich okolí. Ostření se používá ke kompenzaci mírného rozmazání způsobeného optikou fotoaparátu, pohybem nebo pro 
vizuální zvýraznění detailů před tiskem či publikací.