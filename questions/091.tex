\section{Volné užití díla, bezúplatné zákonné license}
Volné užití díla §~30, Bezúplatné zákonné licence §~31-35.

Za užití díla podle tohoto zákona se nepovažuje užití pro osobní potřebu fyzické osoby, jehož účelem není dosažení 
přímého nebo nepřímého hospodářského nebo obchodního prospěchu, nestanoví-li tento zákon jinak. Do práva autorského tak 
nezasahuje ten, kdo pro svou osobní potřebu zhotoví záznam, rozmnoženinu nebo napodobeninu díla.

Zákon pomocí tzv. bezúplatných zákonných licencí také řeší, za jakých okolností a kdo může omezit autorská práva, 
přičemž je velmi důležité, že zákon říká, že tak lze činit pouze v jím určených případech. Ty musí být vždy prováděny 
pouze v odůvodnitelné míře.

Dalším případem bezúplatné zákonné licence je tzv. katalogová licence, která umožňuje použít dílo, pro
propagaci výstavy či prodeje, bez souhlasu autora (§~32), i zde je nutno uvádět všechny náležitosti jako u citací.