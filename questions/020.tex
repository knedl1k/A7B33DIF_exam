\section{Když charakterizujeme barvu z fyzikálního hlediska, představujeme si viditelnou část barevného spektra vlnových 
délek elektromagnetického záření získaného např. rozkladem bílého světla pomocí hranolu (pokus I. Newtona). Napište 
rozsah vlnových délek (od do) v nanometrech [nm], které lidské oko vidí. Uveďte čtyři barvy viditelného spektra 
uspořádané vzestupně podle jejich vlnových délek. (Nápověda: vzpomeňte si na barvy v duze).}