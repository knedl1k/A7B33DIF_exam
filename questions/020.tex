\section{Když charakterizujeme barvu z fyzikálního hlediska, představujeme si viditelnou část barevného spektra vlnových 
délek elektromagnetického záření získaného např. rozkladem bílého světla pomocí hranolu (pokus I. Newtona). Napište 
rozsah vlnových délek (od do) v nanometrech [nm], které lidské oko vidí. Uveďte čtyři barvy viditelného spektra 
uspořádané vzestupně podle jejich vlnových délek. (Nápověda: vzpomeňte si na barvy v duze).}.
\begin{itemize}
  \item
    Viditelné světlo se nachází v rozsahu $400\,\mathrm{nm}$ (modrá) až $750\,\mathrm{nm}$ (červená).
  \item{seřazení podle vlnové délky (vzestupně)\footnotemark{}:\\}
    modrá, tyrkysová, zelená, žlutá, oranžová, červená
\end{itemize}
\footnotetext{Pokud vím, tak Issac Newton nerozlišoval fialovou barvu, proto ji zde neuvádím, jinak se ale řadí před modrou.}
