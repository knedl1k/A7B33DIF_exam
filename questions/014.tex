\section{Srovnejte na konceptuální úrovni z pohledu fotografování vlastnosti dírkové komory a objektivu složeného z 
čoček.}.
\begin{table}[H]
  \center
  \begin{tabular}{|c|c|}
    \hline
    dírková komora & čočka \\\hline
    \hline
    sbírá málo fotonů (je potřeba delší expozice) & sbírá více fotonů \\\hline
    \enquote{Potíže díky ohybu světla na dírce}\footnotemark & je třeba zaostřit \\\hline
    nemá optické vady (zjednodušeně)\footnotemark & trpí optickými vadami, způsobených čočkou \\\hline
  \end{tabular}
\end{table}
\addtocounter{footnote}{-1}
\footnotetext{Citace z presentace, jde o to že když je díra moc velká, obraz bude rozostřený, pokud ale bude moc malá, začnou se projevovat vlnové vlastnosti světla (difrakce), výsledkem tohoto bude opět nejasný obraz.}
\addtocounter{footnote}{1}
\footnotetext{Pokud se velikost otvoru přibližuje tloušťce stěny ve které je, začíná se projevovat vinětace.}
