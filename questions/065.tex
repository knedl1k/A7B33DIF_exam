\section{Huffmanovo kódování}
Pro odstranění redundance při kódování v kompresi dat se používá Huffmanovo kódování. Uveďte jeho myšlenku. Je 
Huffmanovo kódování optimální? Za jakých podmínek? K čemu se používá?

Hlavní myšlenkou Huffmanova kódování je přiřadit častěji se vyskytujícím symbolům kratší kódová slova a méně častým 
symbolům delší kódová slova. Tím se celková délka zakódovaných dat v průměru zmenší.

Ano, Huffmanovo kódování je optimální za podmínky, že se kóduje každý symbol samostatně a jeho pravděpodobnost výskytu 
je předem známá.

Používá se pro bezeztrátovou kompresi, obvykle jako poslední krok třeba v \bb{zip}, \bb{jpeg}, \bb{mp3}.