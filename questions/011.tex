\section{Při pořizování obrazu trojrozměrného (3D) světa kamerou se geometrie zobrazení reprezentuje modelem dírkové 
kamery (tj. perspektivní projekcí), ve kterém se 3D bod \texorpdfstring{$(x,y,z)$}{(x,y,z)} promítne do obrazové roviny 
jako \texorpdfstring{$(x^{\prime},y^{\prime})$}{(x',y')}. Nakreslete odpovídající obrázek (stačí o dimenzi menší, tj. 
plošný). Předpokládejte, že znáte 3D souřadnice \texorpdfstring{$(x,y,z)$}{(x,y,z)}, ohniskovou vzdálenost 
\texorpdfstring{$f$}{f}, tj. vzdálenost obrazové roviny od středu promítání. Odvoďte vztah pro 
\texorpdfstring{$x^{\prime}$}{x'}.}.