\section{Komprese JPEG a blokovací efekt}
\begin{enumerate}[a)]
    \item Vysvětlete princip dnes hojně používané ztrátové metody komprese obrazu podle standardu JPEG? 
    \item Při velkých kompresních poměrech jsou ve výsledku patrné čtverečky rozměru $8 \times 8$. Čím je tento tzv. 
    blokovací efekt způsoben? Proč se k takovému řešení přistoupilo?
\end{enumerate}
a) Komprese JPEG funguje tak, že obraz nejprve rozdělí na malé bloky o velikosti $8 \times 8$ pixelů. V každém bloku 
převede obrazová data na soubor frekvencí. Klíčovým krokem je kvantizace, při které se s velkou přesností zachovají jen 
důležité (nízké) frekvence představující hrubé tvary a barvy, zatímco méně důležité (vysoké) frekvence popisující jemné 
detaily jsou zahozeny nebo silně zjednodušeny. Právě v tomto kroku dochází ke ztrátě dat, ale zároveň k výraznému 
zmenšení velikosti souboru.

b) Blokovací efekt je způsoben tím, že každý blok $8 \times 8$ pixelů je komprimován zcela samostatně, bez ohledu na své 
okolí. Protože sousední bloky o sobě \enquote{neví}, vznikají na jejich hranicích viditelné, ostré přechody, které tvoří 
onu typickou čtvercovou síť. Přistoupilo se na to z výpočetních důvodů.