\section{Interpretace obrazu - pomocí teorie formálních jazyku}
Interpretace (porozumění) obrazu lze matematicky vyjádřit s využitím přístupu teorie formálních jazyků jako 
zobrazení: pozorovaná obrazová data \texorpdfstring{\rightarrow}{→} model teorie. Modelem teorie je konkrétní svět, v 
němž \enquote{teorie} platí. Jedné \enquote{teorii} může odpovídat více různých světů. Interpretaci lze chápat také 
jako zobrazení: syntax \texorpdfstring{\rightarrow}{→} sémantika. Při interpretaci je využívána sémantika, tj. znalost 
o konkrétním světě. V analýze obrazů počítačem obvykle chápeme, že obrazy představují určité objekty. Uveďte dva 
praktické příklady úloh zpracování obrazu, v nichž je interpretace využívána. Jak je interpretace v těchto úlohách 
konkrétně využita?

Rozpoznávání textu (OCR): Počítač vidí obrázek, na kterém jsou pixely tvořící písmena (syntax). Díky znalosti abecedy 
a slov (sémantika) je interpretuje jako konkrétní slova, třeba z naskenované stránky.

Detekce dopravních značek: Kamera auta vnímá barvy a tvary na silnici (syntax). Pomocí uložených informací o dopravních 
značkách a jejich významu (sémantika) interpretuje tyto tvary jako například \enquote{zákaz vjezdu}, což autu umožní 
správně reagovat.